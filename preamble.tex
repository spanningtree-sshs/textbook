%\usepackage[a4paper, margin=3cm]{geometry}
%\usepackage[shortlabels]{enumitem}
%\usepackage{amsmath}
%\usepackage{amsthm}
%\usepackage{amssymb}
%\usepackage{kotex}
%\usepackage{hyperref}
%\usepackage{cleveref}



%%% FONT
\usepackage[finemath]{kotex} % ``한글''
\usepackage{dhucs-nanumfont} % <= 나눔폰트

\usepackage[utf8]{inputenc}

\usepackage[english]{babel}
\usepackage{csquotes}

\usepackage{epigraph} 

\usepackage{titlesec}
\addto\captionsenglish{\renewcommand{\chaptername}{챕터}}
\addto\captionsenglish{\renewcommand{\partname}{단원}}
\addto\captionsenglish{\renewcommand{\figurename}{자료}}

% \titleformat{\section}{\normalfont\Large\bfseries}{\Roman{chapter}.\arabic{section}}{1em}{}
% \renewcommand \thesubsection{\Roman{chapter}.\arabic{section}.\arabic{subsection}}
% \titleformat{\chapter}%
  % {\normalfont\bfseries\Huge}{\Roman{chapter}.}{10pt}{}
%\titleformat{\chapter}{\normalfont\Large\bfseries}{\thechapter}{1em}{}

\addto\captionsenglish{% Replace "english" with the language you use
  \renewcommand{\contentsname}%
    {목차}%
}

%%% BASIC TOOLS
\usepackage{amsmath, amsthm, amssymb, amsfonts, amscd} % Useful math symbols and theorem environments


\usepackage{ifthen} % if statement
\usepackage{color} % Color
\usepackage{mathrsfs} % mathscr
\usepackage{tikz} % Drawing Diagrams
\usepackage{graphicx} % Figures, ...
\usepackage[shortlabels]{enumitem}
\usepackage{imakeidx}


\makeindex[title=찾아보기, intoc]


\usetikzlibrary{cd}

%%% BIBLIOGRAPHY

%%% REFERENCING
\usepackage{hyperref}
\usepackage{cleveref}
\hypersetup{
    colorlinks=true,
    linkcolor=blue}

%%% Theorem and Environments
\theoremstyle{definition}
\newtheorem{theorem}{정리}[section]
\newtheorem{corollary}{따름정리}[theorem]
\newtheorem{lemma}[theorem]{보조정리}
\newtheorem{claim}[theorem]{주장}
\newtheorem*{remark}{참고}
\newtheorem{proposition}[theorem]{명제} %유클리드 원론에서 따옴
\newtheorem*{question}{질문}
\newtheorem{conjecture}[theorem]{추측}
\newtheorem{definition}{정의}[section]
\newtheorem{axiom}[definition]{공리}
\newtheorem*{example}{예시}
\newtheorem{notation}[theorem]{표기법} 
\newtheorem{exercise}{연습문제}[section]
\newtheorem{hint}{힌트}[section]

\crefname{theorem}{정리}{정리}
\crefname{corollary}{따름정리}{따름정리}
\crefname{lemma}{보조정리}{보조정리}
\crefname{claim}{주장}{주장}
\crefname{remark}{참고}{참고}
\crefname{proposition}{명제}{명제}
\crefname{question}{질문}{질문}
\crefname{conjecture}{추측}{추측}
\crefname{definition}{정의}{정의}
\crefname{axiom}{공리}{공리}
\crefname{example}{예시}{예시}
\crefname{notation}{표기법}{표기법}
\crefname{exercise}{연습문제}{연습문제}
\crefname{hint}{힌트}{힌트}
\crefname{equation}{식}{식}
\crefname{figure}{자료}{자료}

\Crefname{theorem}{정리}{정리}
\Crefname{corollary}{따름정리}{따름정리}
\Crefname{lemma}{보조정리}{보조정리}
\Crefname{claim}{주장}{주장}
\Crefname{remark}{참고}{참고}
\Crefname{proposition}{명제}{명제}
\Crefname{question}{질문}{질문}
\Crefname{conjecture}{추측}{추측}
\Crefname{definition}{정의}{정의}
\Crefname{axiom}{공리}{공리}
\Crefname{example}{예시}{예시}
\Crefname{notation}{표기법}{표기법}
\Crefname{exercise}{연습문제}{연습문제}
\Crefname{hint}{힌트}{힌트}
\Crefname{equation}{식}{식}
\Crefname{figure}{자료}{자료}

\crefformat{section}{#2#1절#3}
\Crefformat{section}{#2#1절#3}
\crefformat{chapter}{#2#1장#3}
\Crefformat{chapter}{#2#1장#3}
\crefformat{part}{#2#1단원#3}
\Crefformat{part}{#2#1단원#3}

%%% SYMBOLS
\renewcommand{\qed}{\vspace{0.5em}\hfill////} % End of proof symbol

\newcommand*{\dd}{\mathop{\mathrm{d}\!}}

\DeclareMathOperator{\Aut}{Aut}
\DeclareMathOperator{\Gal}{Gal}
\DeclareMathOperator{\Hom}{Hom}

\DeclareMathOperator{\Domain}{Dom}
\DeclareMathOperator{\Codomain}{Cod}

\DeclareMathOperator{\ArithMean}{AM}
\DeclareMathOperator{\GeomMean}{GM}
\DeclareMathOperator{\HarmMean}{HM}


\DeclareMathOperator{\ImageMap}{Im}
\DeclareMathOperator{\KernelMap}{Ker}
\DeclareMathOperator{\Span}{Span}