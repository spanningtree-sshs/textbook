\chapter{서문}
이 책은 서울과학고등학교의 수학 I 과정의 보충 교재로 쓰여졌다.
즉, 수학 I 과정을 공부하는 학생이 더 높은 과정에서 같은 내용이 어떤 방식으로 다루어지는지를 체험할 수 있도록 하는 것이 우리의 주 목적이였다.
수학 I 과정에서 들 수 있는 많은 의문들 중 `집합이 무엇인가?', `무한집합의 크기는 정의할 수 있을까?', `실수의 제대로 된 정의는 무엇일까?', `다항식은 함수인 것일까?', 와 같은 질문들은 현재 고등교육과정보다 훨씬 더 많은 이론이 필요하고, 우리는 필연적으로 해당 내용을 포함하였다.
그러나 많은 이론이라는 것은 통상적으로 고등 과정의 ``어려움''과는 관련이 적다.

이 이론들의 정리들을 증명하는 과정에서 우리는 각 과목에서 자연스러운 질문이 무엇인지, 그리고 각 수학의 물체들에 대해서 ``제대로'' 생각하는 방법을 알게 된다.
이 책의 많은, 심지어는 대부분의 정리들은 증명 과정이 복잡하지 않다.
그러나 이런 정리들을 쌓아 나가면서, 우리는 시험시간에 우리가 총동원하는 놀라운 기교들과 도구들이 이룰 수 있는 것 보다, 훨씬 더 강력하고 아름다운 수학의 본 모습을 보게 된다.
어떻게 대수적인 성질들과 해석적 성질들이 서로 다르고, 어떻게 서로 혼합하는지 보면서 우리는 ``임의의 $n$차 다항식은 근이 최대 $n$개 있다''나, ``변수 $n$개 방정식을 풀기 위해서 $n$개의 식이 필요하다''와 같은 문장들을 한층 깊게 이해하게 된다.

그러나 이 책에는 많은 이론만 쓰여진 것이 아니다.
몇 개의 신박한 아이디어를 쓰면 풀리는 고립된 문제들과, 이론과는 조금 동떨어져 있지만 유명한 정리들 (e.g. 바나흐-타르스키 역설) 또한 있으며, 순수한 수학의 이해를 주 목적으로 삼지 않는 독자의 흥미를 끌기 위하여 (사실 누가 그렇겠는가?) 다양한 이론들의 놀라운 적용들 까지 포함하려고 노력하였다.
수학 I의 내용이 현대수학의 관점으로 보면 매우 넓은 나머지, 이 책의 분량과 범위 또한 매우 넓어졌다.
한 주제를 깔끔하고 투명하게 설명하는 책을 쓰는 것 부터 매우 어려운데, 기초 집합론에서 해석학, 대수학의 제일 기본적인 부분들을 다루는 좋은 책을 쓰는 것은 거의 불가능한 일일 것이다. (cf. Steenrod, Halmos, Schiffer, Dieudonne, \textit{How to write mathematics}, pp. 21)
우리는 독자가 쉽게 원하는 부분을 읽을 수 있도록 정의들과 정리들을 적절하게 참조하고, 찾아보기를 제공하는 등의 노력을 기울였지만, 분명히 있을 설명상의 부족함과 결함들에 대해, 심심한 사과의 말을 드린다.

동아리 \textit{SpanningTree}에서 진행한 이 프로젝트는, 다인이 참여한 만큼, 각 장의 작가들이 서로 다르다.
이로 인한 표기법이나 용어의 차이를 최소화하려고 노력한 반면, 관점이나 스타일의 차이는 수학을 볼 수 있는 (그리고 설명하는 방법의!) 여러 관점들을 보이기 위하여 보존하였다.

마지막으로, 이 책을 쓰는 데 많은 도움을 준 여러 친구들과 선생님들, 그리고 많은 서적들의 작가들에게 진정한 감사를 표한다.
\medskip
\begin{flushright}
  \textit{SpanningTree}\par
  \today\par
  서울과학고등학교
\end{flushright}