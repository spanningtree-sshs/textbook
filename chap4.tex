\part{방정식과 부등식}

\chapter{절대부등식}
\section{산술 기하 평균 부등식}
\begin{definition}\index{평균}\index{평균!산술}\index{mean}\index{mean!arithmetic}
$n$개의 실수 $a_1, \cdots, a_n$에 대해 이들의 산술평균(arithmetic mean)을 $\ArithMean(a_1, \cdots, a_n)=\frac1n \sum a_i$로 정의한다.
\end{definition}\index{평균!기하}\index{mean!geometric}
$n$개의 음 아닌 실수 $a_1, \cdots, a_n$에 대해 이들의 기하평균(geometric mean)을 $\GeomMean(a_1, \cdots, a_n)=\sqrt[n]{\prod a_i}$로 정의한다. 
\begin{definition}\index{평균!조화}\index{mean!harmonic}
$n$개의 0 아닌 실수 $a_1, \cdots, a_n$에 대해 이들의 조화평균(harmonic mean)을 $\HarmMean(a_1, \cdots, a_n)=1/\ArithMean(1/a_1, \cdots, 1/a_n)$으로 정의한다. 
\end{definition}
\begin{theorem}[산술 기하 평균 부등식]\label{thm:am-gm}\index{산술 기하 평균 부등식}\index{AM-GM inequality}
$n$개의 음 아닌 실수 $a_1, \cdots, a_n$에 대해, $\ArithMean(a_1, \cdots, a_n)\geq \GeomMean(a_1, \cdots, a_n)$이 성립한다. 등호는 정확히 $a_1=\cdots=a_n$일 때만 성립한다. 
\end{theorem}
\begin{exercise}
양수 $a_i$들에 대해 $\GeomMean(a_1, \cdots, a_n)\geq \HarmMean(a_1, \cdots, a_n)$가 성립함을 \cref{thm:am-gm}으로부터 보여라. 
\end{exercise}
여기서는 Cauchy의 forward-backward induction\index{양방향 귀납법}\index{forward-backward induction}을 이용해 \cref{thm:am-gm}을 보일 것이다. 
\begin{exercise}
    $n=2$일 때를 증명하여라. 즉, $a, b\geq 0$에 대해 $\frac{a+b}{2}\geq \sqrt{ab}$임을 보여라. 
\end{exercise}
\begin{exercise}    
    $n=n_1$일 때와 $n=n_2$일 때 \cref{thm:am-gm}이 성립한다면 $n=n_1n_2$일 때도 성립함을 보여라. Hint: ``평균은 평균의 평균이다...''
\end{exercise}
\begin{exercise}
    $n$일 때 성립한다면 $n-1$일 때도 성립함을 보여라. Hint: $a_n=\ArithMean(a_1, \cdots, a_{n-1})$을 대입하자. 
\end{exercise}
\begin{exercise}
    증명을 완료하여라. 
\end{exercise}
여기서, \begin{equation*}
    1/\HarmMean(a_1, \cdots, a_n)=\ArithMean(1/a_1, \cdots, 1/a_n)
\end{equation*}에 주목하자. 잘 생각해 보면 $\GeomMean$에 대해서도 비슷한 식을 세울 수 있다: 
\begin{equation*}
    \log \GeomMean(a_1, \cdots, a_n)=\ArithMean(\log a_1, \cdots, \log a_n)
\end{equation*}
따라서 AM-GM 부등식을
\begin{equation*}
    \log \ArithMean(a_1, \cdots, a_n)\geq \ArithMean(\log a_1, \cdots, \log a_n)
\end{equation*}
으로 다시 쓸 수 있다. 또한 이 부등식이 성립하는 이유는 $\log$의 그래프가 위로 볼록하기 때문이라고 추측해 볼 수 있다. 일반화해 보자. 
\begin{definition}\label{def:convex}\index{볼록함수}\index{convex function}
    함수 $f$가 어떤 구간에서 임의의 실수 $0\leq c\leq 1$에 대해 $$f(ca+(1-c)b)\leq(\geq) cf(a)+(1-c)f(b)$$를 만족할 때 $f$가 그 구간에서 아래로(위로) 볼록하다고, 영어로는 convex(concave)라고 한다. 아래로 볼록한 함수를 간단히 볼록하다고도 한다. 등호가 $c=0, 1$에서만 성립한다면 $f$를 strictly convex(concave)라고 한다. 
\end{definition}
\begin{remark}
    즉, 볼록함수는 할선이 항상 그래프보다 위에 있는 함수이다. 
\end{remark}
\begin{theorem}[젠센 부등식]\label{thm:Jensen}\index{젠센 부등식}\index{Jensen's inequality}
$n$개의 실수 $c_i$가 $c_i\geq 0, \sum c_i=1$을 만족한다고 하자. 아래로 볼록한 함수 $f: \mathbb{R}\to\mathbb{R}$에 대해 $$f\left(\sum c_ia_i\right)\leq \sum c_if(a_i)$$가 성립한다. 특히 $f$가 strictly convex이고 $\forall i, c_i>0$이라면 등호는 $a_1=\cdots=a_n$일 때만 성립한다. 
\end{theorem}
\begin{exercise}
\cref{thm:am-gm}과 같은 방식으로 증명할 수 있다. 증명하여라. 기하적으로도 설명해 보라. 
\end{exercise}
\begin{remark}
    이 정리에 $c_i=1/n, f=-\log$를 대입하면 \cref{thm:am-gm}을 얻는다. 
\end{remark}
\begin{remark}
    확률변수 $X$와 볼록함수 $f$에 대해 젠센 부등식의 다른 형태 $f(\mathbb{E}[X])\leq \mathbb{E}[f(X)]$도 짚고 넘어가자. 
\end{remark}
다음은 기하평균의 성질을 표현하는 다른 부등식이다.
\begin{theorem}[Carleman]
\label{thm:carlemanineq} \index{Carleman's inequality} \index{카를만의 부등식}
    모든 $0 \leq a_n$과 $N$에 대해, 다음이 성립한다.
    \begin{equation*}
        \sum_{n = 1}^N \GeomMean(a_1, \dots, a_n) \leq e \sum_{n = 1}^N a_n
    \end{equation*}
\end{theorem}

\begin{exercise}
    다음은 \Cref{thm:carlemanineq}의 한 증명이다.
    먼저, $(1 + 1/m)^m \leq e$가 모든 자연수 $m$에 대해서 성립한다는 사실은 자유롭게 사용하여도 된다.

    \begin{enumerate}[(a)]
        \item \Cref{thm:am-gm}을 사용하여 바로 증명을 시도해보아라.
        \item 위 증명이 안되는 이유는 \Cref{thm:am-gm}는 $a_1, a_2, \dots, a_n$의 크기가 다를 때, 잃는 것이 많기 때문이다.
        즉 이 문제를 해결하기 위해 적당히 선택된 $c_m$에 대해 $a'_m = a_m c_m$로 놓고 증명을 마쳐라. [\textit{힌트}: $(c_1 c_2 \dots c_n)^{-1/n} n^{-1}$이 $e$에 가깝도록 선택을 해 보아라.]
    \end{enumerate}
\end{exercise}

\section{\texorpdfstring{$\mathbb{C}^n$ 공간 위 부등식}{C\textasciicircum n 공간 위 부등식}}
\label{sec:cninequality}
자연수 $n$에 대해서, $\mathbb{C}^d$ 집합 위의 다음 함수들을 고려하자.
\begin{align}
\label{eq:pnormdef}
\begin{split}
    \|x\|_p &= \left( \sum_{i = 1}^d |x_i|^p \right)^{1/p} \quad (1 \leq p < \infty) \\
    \|x\|_\infty &= \max_{1 \leq i \leq d} |x_i|
\end{split}
\end{align}
이 때 지수 $p$는 $\infty$의 값 또한 가질 수 있고, $1/\infty = 0$으로 생각할 것이다.

\begin{definition}
    집합 $\mathbb{C}^d$과 $\|x\|_p$를 묶어서, 간편하게 공간 $L^p_d$으로 표기한다.
\end{definition}
\begin{remark}
    공간 $L^p$를 따로 정의하는 것은, 우리의 부등식 대부분이 무한수열이나, $(-\infty, \infty)$위의 적당한 함수에 대해 성립하기 때문이다.
    이 때에는 수렴성이 중요해지기 때문에, 적당한 의미에서 $|x|^r$의 합 (적분 또는 급수)가 수렴하는 수학적인 물체들의 집합을 $L^r$로 표기한다.
    우리는 이 전통을 따라, $\mathbb{C}^d$의 경우에서도 $L^r, L^p$등으로 따로 그 지수를 표기할 것이다.
\end{remark}
\begin{exercise}
    \Cref{eq:pnormdef}의 함수가 모든 $1 \leq p \leq \infty$와, $c \in \mathbb{C}, x \in L^p_d$에 대해, $\|cx\|_p = |c|\|x\|_p$를 만족함을 보이시오.
    또한, $\|x\|_p = 0$과 $x = 0$이 필요충분조건임을 보이시오.
\end{exercise}

\begin{definition}
\label{def:lpspace1}
    두 $x, y \in \mathbb{C}^d$에 대해, $xy = (x_i y_i) \in \mathbb{C}^d$으로 정의한다.
\end{definition}
\begin{lemma}
\label{lem:ineqyoung1}
    실수 $1 < p, q$가 $1/p + 1/q = 1$이라고 가정하자.
    두 복소수 $x, y$에 대해, $|xy| \leq |x|^p/p + |y|^q/q$이 성립한다.
\end{lemma}
\begin{exercise}
    \Cref{lem:ineqyoung1}을 증명하시오. [\textit{힌트}: 자연로그 $\log$는 $(0, \infty)$에서 위로 볼록하다.]
\end{exercise}

다음 두 부등식은 널리 알려져 있다.
자신 있는 독자는 증명을 먼저 시도해보는 것도 좋을 것이다.

\begin{theorem}[H\"older]
\label{thm:holderineq}
    두 $1 \leq p, q \leq \infty$가 $1/p + 1/q = 1$을 만족한다고 하자.
    그러면 $x \in L^p_d, y \in L^q_d$에 대해, $\|xy\|_1 \leq \|x\|_p \|y\|_q$가 성립한다.
    \index{H\"older's inequality}\index{횔더 부등식}
\end{theorem}
\begin{proof}
    \Cref{lem:ineqyoung1}에 $x_i/\|x\|_p, {y_i}/\|y\|_q$를 대입하면,
    \begin{equation*}
        \frac{|x_i {y_i} |}{\|x\|_p \|y\|_q} \leq \frac{1}{p} \left(\frac{|x_i|}{\|x\|_p}\right)^p + \frac{1}{q} \left(\frac{|{y_i}|}{\|y\|_q}\right)^q
    \end{equation*}
    를 얻고, 모든 $i$에 대해서 양변을 더하면,
    \begin{equation*}
        \frac{\|xy\|_1}{\|x\|_p \|y\|_q} \leq \frac{1}{p} + \frac{1}{q}
    \end{equation*}
    에서 증명이 끝났다.
\end{proof}

\begin{exercise}[Generalized H\"older]
\label{exer:genholder}\index{H\"older's inequality!generalized}\index{횔더 부등식!의 일반화}
    세 $1 \leq p, q, r \leq \infty$에 대해서,
    \begin{equation*}
        \frac{1}{p} + \frac{1}{q} = \frac{1}{r}
    \end{equation*}
    이 성립하면, 두 $x \in L^p_d, y \in L^q_d$에 대해 $\|xy\|_r \leq \|x\|_p\|y\|_q$를 증명하여라.
    [\textit{힌트}: $0 \leq p \leq 1$이고 $a, b$가 양수이면 $a^pb^{1 - p} \leq p a + (1 - p) b$이 성립한다. 이를 이용해 $\|x\|_p = \|y\|_q = 1$인 경우를 고려해라.]
\end{exercise}

\begin{theorem}[Minkowski]
\label{thm:minkowineq}
    임의의 $1 \leq p \leq \infty$에 대해서, $\|x + y\|_p \leq \|x\|_p + \|y\|_p$가 성립한다.
    \index{Minkowski's inequality}\index{민코프스키 부등식}
\end{theorem}
\begin{proof}
    만약 $p = 1$이거나 $p = \infty$이면 매우 쉽다.
    즉 $1 < p < \infty$임을 가정하자.

    삼각부등식을 $|x_i + y_i|$에 적용하면
    \begin{equation*}
        |x_i + y_i|^p \leq |x_i| |x_i + y_i|^{p - 1} + |y_i| |x_i + y_i|^{p - 1}
    \end{equation*}
    을 얻고, 모든 $i$에 대해 다 더한 뒤, 각 항에 횔더 부등식을 $1/p + 1/q = 1$인 $q$에 적용하자.
    \begin{equation*}
        \|x + y\|^p_p \leq (\|x\|_p + \|y\|_p) \left(\sum |x_i+y_i|^{q (p - 1)} \right)^{1/q} 
    \end{equation*}
    이 때, $q(p - 1) = p$이므로, 다시 이 식을 쓰면,
    \begin{equation*}
        \|x + y\|^p_p \leq (\|x\|_p + \|y\|_p) (\|x + y\|_p)^{p/q}
    \end{equation*}
    이므로, 양변을 $\|x + y\|^{p/q}_p$로 나누면 증명이 끝났다.
\end{proof}

\begin{theorem}[H\"older, Converse]
\label{thm:holderconv}
    지수 $1 \leq p, q \leq \infty$가 $1/p + 1/q = 1$을 만족한다고 가정하자.
    만약 모든 $y \in L^q_d$에 대해서, $\|xy\|_1 \leq C \|y\|_q$가 성립시, $\|x\|_p \leq C$가 성립한다.
    \index{H\"older's inequality!converse} \index{횔더 부등식!역}
\end{theorem}
\begin{exercise}
    \Cref{thm:holderconv}를 증명하시오.
\end{exercise}

서로 다른 $p_1, p_2$에 대해서, $\|x\|_{p_1}, \|x\|_{p_2}$의 연관을 물어볼 수 있다.
즉 $x$를 고정하고, $\|x\|_p$를 $p$의 함수로 보는 관점 또한 유효하다.
\begin{theorem}
\label{thm:pnormint1}
    두 지수 $1 \leq p_1 < p_2 \leq \infty$에 대해서, 아무 $0 \leq t \leq 1$에 관해 지수 $1/p = t/p_1 + (1 - t)/p_2$로 정의하자.
    그러면 모든 $x \in \mathbb{C}^d$에 대해,
    \begin{equation*}
        \|x\|_p \leq \|x\|^{t}_{p_1} \|x\|^{1 - t}_{p_2}
    \end{equation*}
    가 성립한다.
\end{theorem}
\begin{proof}
    만약 $p_2 = \infty$이면 $|x_i|^p \leq |x_i|^{p_1} \|x\|_\infty^{p - p_1}$로 쉽게 증명 가능하다.
    즉 $p_2 < \infty$인 경우를 해보자.
    우리는 어떤 상수 $0 \leq \alpha \leq 1$에 대해 $p = \alpha p_1 + (1 - \alpha) p_2$임을 알고, 우리가 했듯이 로그가 위로 볼록한 사실을 사용해, 다음을 보일 수 있다.
    \begin{equation*}
        \alpha \log |x_i|^{p_1} + (1 - \alpha) \log |x_i|^{p_2} \leq \log (\alpha |x_i|^{p_1} + (1 - \alpha)|x_i|^{p_2} )
    \end{equation*}
    여기에서 식을 정리하면,
    \begin{equation}
    \label{eq:pnormint1}
        |x_i|^p \leq \alpha |x_i|^{p_1} + (1 - \alpha) |x_i|^{p_2}
    \end{equation}
    을 볼 수 있고, $x_i$를 모두 다 더하면, $\|x\|_{p_1} = \|x\|_{p_2} = 1$인 경우에 해당 부등식을 얻을 수 있다.
    더욱 더 나아가, 양변에 $0 < C$를 곱하면, 우리의 결론은 $\|x\|^{p_1}_{p_1} = \|x\|^{p_2}_{p_2} = C$일 때에도 성립한다.
\end{proof}

\begin{exercise}
    \Cref{thm:pnormint1}의 증명을 완료하시오. [\textit{힌트}: \Cref{eq:pnormint1}에서, $x_i$ 대신 $\beta x_i$을 고려해라.]
\end{exercise}

\begin{exercise}
    \Cref{thm:pnormint1}를, $|x|^{(1 - t)p + tp}$와 횔더 부등식을 사용해서 증명하시오.
\end{exercise}

이제 두 $\mathbb{C}^d$에 대한 합성곱 연산을 정의할 것이다.
\begin{definition}
\label{def:finconv}
    두 $x, y \in \mathbb{C}^d$의 합성곱(Convolution) \index{합성곱}\index{convolution} $x * y$는 다음과 같이 정의한다.
    \begin{equation*}
        (x * y)(i) = \sum_{1 \leq j \leq n} x(j)y(i - j)
    \end{equation*}
    여기서 $i$와 $j$는 $\mathbb{Z}/n\mathbb{Z}$, 즉 $i, j \mod{n}$으로 본다.
\end{definition}
\begin{remark}
    우리는 앞으로 간편함을 위해, 특별한 말이 없으면 $\sum$의 인덱스를 $1$에서 $d$까지, 그리고 모든 $x\in \mathbb{C}^d$에 대해서, $x(n)$은 $x(n \pmod{d})$로 해석할 것이다.
\end{remark}

\begin{exercise}
    벡터 $x, y, z \in \mathbb{C}^n$에 대해, $x * y = y * x$와 $(x * y) * z = x * (y * z)$를 증명하시오.
    또한, $a, b \in \mathbb{C}$에 대해, $(ax + by) * z = a (x * z) + b (y * z)$를 증명하시오.
\end{exercise}

\begin{example}
    복소수 $\omega$가 $\omega^n = 1$를 만족한다고 가정하자.
    그러면 $n$차 복소다항식 $p(x), q(x)$을 각각
    \begin{align*}
    \begin{split}
        p(x) &= p_0 + p_1x + \dots + p_{n - 1}x^{n - 1} \\
        q(x) &= q_0 + q_1x + \dots + q_{n - 1}x^{n - 1}
    \end{split}
    \end{align*}
    로 정의하고, $x(i) = p_i, y(i) = q_i$로 정의하면 (여기서 $0 = n$으로 생각한다),
    \begin{equation*}
        p(\omega)q(\omega) = (x * y)(0) + (x * y)(1)\omega + \dots + (x * y)(n - 1)\omega^{n - 1}
    \end{equation*}
    이 성립한다.
\end{example}

\begin{example}
    면이 $n$개이고, 각각 $0$에서 $n - 1$이 적혀있는 주사위 $A$에서 $i$가 쓰여진 면이 나올 확률이 $a_i$이고, $B$에서 마찬가지로 $b_i$이라고 하자.
    주사위 $A, B$를 나타내는 $a, b \in \mathbb{C}^d$을 $a = (a_i), b = (b_i)$로 정의하면, $(a * b)(i)$는, 각각 $A, B$를 굴려 나온 값의 합이 $i \pmod{n}$일 확률이다.
\end{example}

합성곱 연산에 대한 다음 부등식을 증명할 것이다.
\begin{theorem}[Young]
\label{thm:youngconvineq} \index{Young's convolution inequality}\index{영의 합성곱 부등식}
    지수 $1 \leq p, q, r \leq \infty$가 다음을 만족하면,
    \begin{equation*}
        \frac{1}{p} + \frac{1}{q} = 1 + \frac{1}{r}
    \end{equation*}
    두 $x \in L^p_d, y \in L^q_d$에 대해, $\|x * y\|_r \leq \|x\|_p \|y\|_q$가 성립한다.
\end{theorem}

여기에 증명이 하나 있다. (필자가 추천하지는 않는 증명이다.)
\begin{proof}
    만약 $r = \infty$이면 횔더 부등식에서 바로 유도가 가능하다.
    즉 $r < \infty$라고 가정 가능하다.

    먼저 $(x * y)(n)$을 다음과 같이 쓰자.
    \begin{equation*}
        |(x * y)(n)| = \sum_m |x(m)|^{\frac{r}{r + 1}} |y(n - m)|^{\frac{1}{r + 1}} |x(m)|^{\frac{1}{r + 1}} |y(n - m)|^{\frac{r}{r + 1}}
    \end{equation*}
    나누어진 두 부분에 횔더 부등식을 적용하자.
    우리의 지수에 대한 조건을
    \begin{equation*}
        \frac{1}{\frac{r + 1}{r} p} + \frac{1}{\frac{r + 1}{r} q} = 1
    \end{equation*}
    으로 쓰면, 횔더 부등식의 조건을 만족함을 볼 수 있고, 적용시
    \begin{equation*}
        |(x * y)(n)| \leq \left( \sum_m |x(m)|^p |y(n - m)|^{\frac{p}{r}}\right)^{\frac{r}{(r + 1)p}} \left( \sum_m |x(m)|^{\frac{q}{r}} |y(n - m)|^q\right)^{\frac{r}{(r + 1)q}}
    \end{equation*}
    첫번째 항에서 $|y|^q$ 꼴을 만들기 위해 지수 $\alpha = rq/p$를 준비한다.
    그리고 $|x|^p$를 유지하기 위해, $1/\alpha + 1/\alpha' = 1$이라고 할 때, $|x|^p$를 각각 $|x|^{\frac{p}{\alpha}}|x|^{\frac{p}{\alpha'}}$로 쪼갠다.
    다시 횔더 부등식 지수 $\alpha, \alpha'$에 대해 적용하면,
    \begin{equation*}
        \left( \sum_m |x(m)|^p |y|^{\frac{p}{r}}(n - m)\right) \leq \left(\sum_m |x(m)|^p \right)^{\frac{1}{\alpha'}} \left(\sum_m |x(m)|^p |y(n - m)|^q \right)^{\frac{1}{\alpha}}
    \end{equation*}
    을 얻는다.
    비슷한 방법으로 두번째 항을 $\beta = rp/q$로 정리하고, $x, y$의 단독항을 $\|x\|_p, \|y\|_q$로 쓰면,
    \begin{equation*}
        |(x * y)(n)| \leq \left( \sum_m |x(n)|^p |y(m - n)|^q \right)^{\frac{1}{r + 1} (\frac{1}{p} + \frac{1}{q})} \|x\|_p^{\frac{r}{r + 1}(1 - \frac{p}{qr})} \|y\|_q^{\frac{r}{r + 1}(1 - \frac{q}{pr}}
    \end{equation*}
    을 얻는다.
    양쪽을 $r$승한 뒤에, 모든 $n$에 대해서 더하면, 원하는 부등식이 나온다.
\end{proof}

\begin{exercise}
    버튼을 누르면 $1, \dots, n$중 하나를 각각 $p_1, \dots, p_n$의 확률로 출력하는 기계가 있다.
    이 기계의 버튼을 두번 눌러, 그 합을 $n$으로 나눈 나머지를 얻는 시행을 시행 $A$라고 하자.
    만약 $p_i < t$가 어떤 $1/n < t < 1$에 대해, 모든 $i$에 대해 성립시, 시행 $A$를 $m$번 하여 나온 값이 모두 같을 확률이 $t^{m - 1}$ 이하임을 증명하시오.
    [\textit{힌트}: 시행 1회당 오직 하나가 일어나는 사건들의 벡터를 $x$라고 하자. 그러면 $\|x\|^p_p$는 $x$를 $p$번 독립시행시, 같은 사건이 $p$번 일어날 확률이다. \Cref{thm:youngconvineq}를 사용하시오.]
\end{exercise}

\begin{exercise}
    \Cref{thm:youngconvineq}를 다음과 같이 일반화 하시오.
    만약, $1 \leq i \leq n$ 각각에 대해, $x_i \in L^{p_i}_d$가 존재하고, 지수들이 다음 식을 만족하면,
    \begin{equation*}
        \sum_{i = 1}^n \frac{1}{p_i} = n - 1 + \frac{1}{r}
    \end{equation*}
    부등식 $\|x_1 * x_2 * \dots * x_n\|_r \leq \|x_1\|_{p_1}\|x_2\|_{p_2}\dots \|x_n\|_{p_n}$ 이 성립한다.
\end{exercise}

\section{\texorpdfstring{$L^p$}{L\textasciicircum p} 공간 위 부등식}
우리는 지금까지 \Cref{def:lpspace1}에서 성립하는 매우 다양한 부등식들을 보았다.
그러나 우리는 복소수의 중요한 성질들을 쓰기 보다는, 절댓값 함수의 몇가지 중요한 성질들만을 사용하여 모든 식을 이끌어 내었다.
즉, 우리는 지금까지 사용했던 정의를 살짝 추상적으로 바꾸어 손쉽게 더 많은 부등식들을 얻을 수 있다.

\begin{definition}
    실수나, 복소수 위 정의된 벡터공간 $V$위의 함수 $\|\cdot\| : V \to \mathbb{R}$가 다음을 만족하면, 우리는 이 함수를 세미노름이라고 부른다.
    \begin{enumerate}[(a)]
        \item 모든 벡터 $x \in V$에 대해, $0 \leq \|x\|$이다.
        \item 모든 벡터 $x \in V$와 (실수 또는 복소수인) 스칼라 $\alpha$에 대해, $\|\alpha x\| = |\alpha| \|x\|$이다.
        \item 모든 벡터 $x, y \in V$에 대해, 삼각부등식 $\|x + y\| \leq \|x\| + \|y\|$이 성립한다.
    \end{enumerate}
    벡터공간 $V$와 세미노름 $\|\cdot\|$을 묶어 세미노름벡터공간이라고 부른다.
\end{definition}

\begin{definition}
    집합 $\{1, \dots, d \} = I_d$에서, 음이 아닌 실수들로 가는 함수 $\mu$를 우리는 가중치함수라고 한다.
    가중치에 대한 $a_m$의 급수를 다음과 같이 정의한다.
    \begin{equation*}
        \sum_m a_m \mu(m)
    \end{equation*}
\end{definition}

\begin{definition}
\label{def:lpspace2}
    정의역 집합 $\{1, \dots, d\} = I_d$과, 치역인 세미노름벡터공간 $V$, 그리고 가중치함수 $\mu$에 대해, 모든 $f: I_n \to V$의 함수들의 벡터 공간에, 각 지수 $1 \leq p \leq \infty$에 대해서 다음과 같은 세미노름을 부여하자.
    \begin{align}
    \begin{split}
        \|x\|_p &= \left( \sum_{i = 1}^d \|x_i\|^p \mu(i) \right)^{1/p} \quad (1 \leq p < \infty) \\
        \|x\|_\infty &= \max_{1 \leq i \leq d} \|x_i\|
    \end{split}
    \end{align}
    이 때, $\|\cdot\|_p$을 부여한 이 세미노름벡터공간을 우리는 $L^p_d(\mu; V)$로 부른다.
\end{definition}

\begin{remark}
    세미노름과 노름의 정의의 차이는, 노름은 $\|x\| = 0 \iff x = 0$이라는 조건이 있으나, 세미노름은 이 조건이 없다.
    우리가 세미노름을 취한 이유는 가중치 함수 $\mu$가 어떤 점에서 $0$인 경우, $\|x\| = 0$이나 $x \neq 0$ 일 수 있기 때문이다.
    현대 적분론 그리고 측도론에서는 이 문제를 해결하기 위해, 각 함수들이 아닌, ``가중치 함수''가 $0$을 부여하는 집합에서만 서로 다른 함수들의 동치관계를 생각한다.
    예를 들어, $[0, 1]$의 적분 가능한 복소함수들의 공간 $L^1[0, 1]$에서, $0$에서 $1$이고 나머지 부분에서 $0$인 함수와, 그냥 영함수는 같은 원소이다.
    그렇기 때문에 $\mu$-a.e., 즉 가중치 $\mu$가 $0$일 수 있는 부분을 제외하고 라는 뜻의 표현을 자주 볼 수 있다.
    그러나 우리는 $L^p$공간을 자세하게 다룰 것이진 않을 것이므로, 간편성을 위해, 노름이라는 조건을 세미노름으로 약화시켰다.
\end{remark}

\begin{example}
    가중치함수 $\mu$를 $\mu(i) = 1$로 정의하고, 치역을 $\mathbb{C}$로 하면, \Cref{def:lpspace1}의 $L^p_d$ 공간을 얻을 수 있다.
    일반적으로, $\mu(i) = 1$인 가중치 함수를 가진 $L^p_d(\mu; V)$ 공간을 우리는 $L^p_d(V)$로 축약해 쓴다.
    비슷하게, $V = \mathbb{C}$이면, $L^p_d(\mu; V)$ 공간을 우리는 $L^p_d(\mu)$로 축약해 쓴다.
\end{example}

\begin{remark}
    가중치함수는 엄밀한 의미의 일반화는 아니지만, (cf. \Cref{exer:lpgenholder}), 몇가지 증명을 더 간단하게 할 수 있다.
    예를 들어, \Cref{thm:youngconvineq}의 $|x|^p$를 쪼개는 과정을, 가중치함수를 $\mu'(m) = |x(m)|\mu(m)$으로 변화하는 과정으로 더 쉽게 볼 수 있다.
\end{remark}

\begin{example}
    $e_i \in \mathbb{C}^n$를 $i$번째 성분이 $1$이고 나머지 성분이 $0$인 벡터로 정의하고, 각 $i$에 대해, $f_i = e_1 + \dots + e_i$로 정의하자.
    그러면 함수 (또는 순서쌍) $f = i \mapsto f_i$는 공간 $L^p_d(L^p_d)$의 원소이다.
    이 벡터의 노름은
    \begin{equation*}
        \|f\|_p = \left( \sum_m \|f_m\|^p_p \right)^{\frac{1}{p}} = \left( \sum_m \sum_l |f_{m, l}|^p \right)^{\frac{1}{p}}
    \end{equation*}
    으로, $(d(d + 1)/2)^{(1/p)}$이다.
\end{example}

\begin{exercise}
    다음을 $x \in L^p_d(\mu;V)$에 대해 증명하시오.
    \begin{equation*}
        \| \sum_m x(m) \mu(m) \| \leq \sum_m \|x(m)\| \mu(m)
    \end{equation*}
\end{exercise}

이 일반적인 경우에 대해서도 같은 부등식들을 증명할 수 있다.

\begin{theorem}[Generalized H\"older, $L^p$]
\label{thm:lpholder} \index{H\"older's inequality} \index{횔더 부등식}
    지수 $1 \leq p, q, r \leq \infty$가
    \begin{equation*}
        \frac{1}{p} + \frac{1}{q} = \frac{1}{r}
    \end{equation*}
    를 만족하고, $x \in L^p_d(\mu), y \in L^q_d(\mu; V)$이면, $\|xy\|_r \leq \|x\|_p\|y\|_q$이다.
    여기서 $xy$는, 각 $i$에 대해 $x(m)y(m)$의 값을 가지는 $L^r_d(\mu; V)$의 원소이다.
\end{theorem}
\begin{exercise}
\label{exer:lpgenholder}
    \Cref{exer:genholder}와 같은 방법으로 증명하시오. [\textit{힌트}: $|\mu(n)|^r = |\mu(n)|^{r/p}  |\mu(n)|^{r/q}$]
\end{exercise}

\begin{theorem}[H\"older, $L^p$ Converse]
\label{thm:holderconvlp}
    지수 $1/p + 1/q = 1, 1 \leq p,q \leq \infty$이라고 하자.
    만약 $x \in L^p_d(\mu)$가 모든 $y \in L^q_d(\mu)$에 대해 $|\sum x(n) y(n) \mu(n)| \leq C \|y\|_q$이면, $\|x\|_p \leq C$이다.
\end{theorem}

\begin{exercise}
    \Cref{thm:holderconvlp}를 증명하시오.
\end{exercise}

\begin{theorem}[Minkowski, $L^p$]
\label{thm:lpminkowski} \index{Minkowski's inequality} \index{민코프스키 부등식}
    지수 $1 \leq p \leq \infty$와, $x, y \in L^p(\mu; V)$에 대해 $\|x + y\|_p \leq \|x\|_p + \|y\|_p$가 성립한다.
\end{theorem}
\begin{exercise}
    \Cref{thm:minkowineq}와 같은 방법으로 증명하시오.
\end{exercise}

\begin{lemma}
\label{lem:genminkowski}
    벡터 $K \in L^1_{d_1}(\mu_1; L^p_{d_2}(\mu_2; V))$를, $K(m, n) = (K(m))(n)$으로, 즉 $K(m) \in L^p_{d_2}(\mu_2; V)$의 $n$번째 성분을 $K(m, n)$으로 표기하자.
    만약 $1 \leq p \leq \infty$이면, 다음이 성립한다.
    \begin{equation*}
        \left( \sum_n \left( \sum_m |K(m, n)| \mu_1(m) \right)^p \mu_2(n) \right)^{1/p} \leq \sum_m \left( \sum_n |K(m, n)|^p \mu_2(n)\right)^{1/p} \mu_1(m)
    \end{equation*}
    더 축약하여, $K(n)$을 $K(m, n)$의 $m$에 대한 함수로, $L^1_{d_1}(\mu_1; V)$의 원소로 취급하면, 다음이 성립한다.
    \begin{equation*}
        \left( \sum_n \|K(n)\|^p_1 \mu_2(n) \right)^{1/p} \leq \sum_m \|K(m)\|_p \mu_1(m)
    \end{equation*}
\end{lemma}
\begin{proof}
    이 식의 우변은 다음과 같이 쓸 수 있다.
    \begin{equation*}
        \sum_m \|K(m)\|_p \mu_1(m)
    \end{equation*}
    그러나 \Cref{thm:lpminkowski}에 의해, 다음이 성립한다.
    \begin{equation*}
        \left\| \sum_m K(m) \mu_1(m) \right\|_p \leq \sum_m \|K(m)\|_p \mu_1(m)
    \end{equation*}
    좌변 안의 급수의 $n$번째 성분은 $\sum_m K(m, n) \mu_1(m)$과 같다.
    즉 여기에 $\|\cdot\|_p : L^p_{d_2}(\mu_2; V) \to \mathbb{R}$의 정의를 사용하면
    \begin{equation*}
        \left( \sum_n \left| \sum_m K(m, n) \mu_1(m) \right|^p \mu_2(n) \right)^{1/p}
    \end{equation*}
    가 된다.
    절댓값을 안에 넣기 위해서는, $K(m, n)$의 모든 원소에 절댓값을 미리 취해준 $|K(m, n)| \in L^q_{d_1}(\mu_1; L^p_{d_2}(\mu_2; \mathbb{R}))$에 이 정리를 적용하면, 원했던 식을 얻는다.
\end{proof}

\begin{theorem}[Generalized Minkowski, $L^p$]
\label{thm:genminkowski} \index{Minkowski's inequality!generalized} \index{민코프스키 부등식!일반화된}
    지수 $1 \leq p, q \leq \infty$가 $ p \leq q$을 만족하고, \Cref{lem:genminkowski}의 표기법에 따른 $K \in L^q_{d_1}(\mu_1; L^p_{d_2}(\mu_2; V))$와, $K'(n, m) = K(m, n)$으로 정의된 $K' \in L^p_{d_2}(\mu_2; L^q_{d_1}(\mu_1; V))$에 대해, $\|K'\|_p \leq \|K\|_q$가 성립한다.
    더 자세히 쓰면,
    \begin{equation*}
    \begin{split}
        &\left( \sum_m \left( \sum_n |K(m, n)|^p \mu_2(n) \right)^{q/p} \mu_1(m) \right)^{1/q}  \\
        &\leq \left( \sum_n \left( \sum_m |K(m,n)|^q \mu_1(m) \right)^{p/q} \mu_2(n) \right)^{1/p}
    \end{split}
    \end{equation*}
    가 성립한다.
\end{theorem}
\begin{proof}
    지수 $k = q/p$와 벡터 $|K|^p$를 \Cref{lem:genminkowski}에 적용해라.
\end{proof}

\begin{remark}
    \Cref{thm:genminkowski}는 $0 < p \leq q$일때 또한 성립하나, 우리는 $0 < p < 1$인 지수는 고려하지 않을 것이다.
\end{remark}

다음 정의는 특히 선형인 부등식에서, 결과를 쉽게 정리하는데 도움을 준다.

\begin{definition}
\label{def:opnorm} \index{operator norm}\index{연산자 노름}
    두 세미노름벡터공간 $V, W$에 대해서, $T: V \to W$가 모든 $x \in V$에 대해 $\|Tx\| \leq C\|x\|$가 성립하도록 하는 $C$의 최솟값을 $T$의 (존재하면) 연산자 노름이라고 하고, $\|T\|$로 표기한다.
\end{definition}

\begin{example}
    벡터 $x \in L^q_d$에 대해서, $T : L^p_d \to \mathbb{C}$를 $T(y) = \sum x(n)y(n)$으로 정의하면, $\|T\| = \|x\|_q$이다. (\Cref{thm:holderineq}, \Cref{thm:holderconv})
\end{example}
\begin{example}
    지수 $1 \leq p \leq \infty$에 대해서 $1/p + 1/q = 1 + 1/r$이라고 하자.
    벡터 $y \in L^p_d$를 고정하고, $x \in L^q_d$에 대해 $T: L^q_d \to L^r_d$를 $(Tx)(n) = (y * x)(n)$으로 정의하면, $\|T\| \leq \|y\|_p$이다. (\Cref{thm:youngconvineq})
\end{example}

이 예제를 다음과 같이 일반화할 수 있다.
\begin{theorem}
\label{thm:kernelop1}
    벡터 $K \in L^{p'}_d(\mu_1; L^q(\mu_2))$를 고정해 각 $x \in L^p(\mu_2)$에 대해서, $T: L^p(\mu_2) \to L^q(\mu_2)$를 다음과 같이 정의하자.
    \begin{equation*}
        (T(x))(m) = \sum_n K(m, n) x(n) \mu_2(n)
    \end{equation*}
    만약 $1/p' + 1/p = 1$이면, $\|T\| \leq \|K\|_{p'}$이다.
\end{theorem}
\begin{exercise}
    \Cref{thm:kernelop1}을 증명하시오.
\end{exercise}

기타 $L^p$의 부등식을 하나 더 살펴보자.

\begin{theorem}[Hardy-Littlewood-Polya]
\label{thm:hdypolyaineq}
    지수 $1 < p < \infty$와, $1/p + 1/q = 1$인 $q$를 고정하자.
    $1$이상의 자연수들의 쌍에 정의된 함수 $K : \mathbb{N}_{+}^2 \to \mathbb{R}$이 $0 \leq K$이고, $K(sa, sb) = s^{-1}K(a, b)$을 만족하며, $K(x, 1)x^{-1/p}, K(1, y)y^{-1/q}$이 $x, y$에 대해 (단조)감소함수이고
    \begin{align*}
        \int_0^\infty K(x, 1) x^{-1/p} dx &= A \\
        \int_0^\infty K(1, y) y^{-1/q} dy &= A'
    \end{align*}
    이면, 모든 $N, M$과, $x \in L^p_N(\mu_1), y \in L^q_M(\mu_2)$에 대해,
    \begin{equation*}
        \left| \sum_m \sum_n K(m,n)x(m)y(n) \mu_2(n) \mu_1(m) \right| \leq A^{1/p} A'^{1/q} \|x\|_p \|y\|_q
    \end{equation*}
    가 성립한다.
    
\end{theorem}

\begin{exercise}
    다음은 \Cref{thm:hdypolyaineq}의 증명이다.
    \begin{enumerate}[(a)]
        \item 모든 $n, N, M$에 대해, 다음이 성립함을 보여라.
        \begin{align*}
            \sum_{k = 1}^N K\left(\frac{k}{n}, 1\right) \left(\frac{k}{n}\right)^{-1/p} \frac{1}{n} &\leq A \\
            \sum_{k = 1}^M K\left(1, \frac{k}{n}\right) \left(\frac{k}{n}\right)^{-1/q} \frac{1}{n} &\leq A'
        \end{align*}
        \item 공간 $L^p_{m \times n}(\mu, V)$를 \Cref{def:lpspace2}와 같지만, 정의역을 $I_m \times I_n$으로 하도록 정의하고, 
        \begin{equation*}
            \mu(m, n) = |K(m, n)| \mu_1(m)\mu_2(n)
        \end{equation*}
        으로 정의하면,
        \begin{equation*}
            \left| \sum_{m, n} x(m)y(n) \mu(m, n) \right| \leq A^{1/p} A'^{1/q} \|x\|_p \|y\|_q
        \end{equation*}
        임을 증명하면 충분함을 보여라.
        \item 급수 안의 $x(m)y(n)$을 $x(m)(m/n)^{1/pq} y(n)(n/m)^{1/pq}$으로 정의한 후, 횔더 부등식을 써서, 증명을 마쳐라.
    \end{enumerate}
\end{exercise}

\begin{theorem}[Hilbert's Double Series]
\label{thm:hilbertineq} \index{Hilbert's double series inequality} \index{힐베르트의 이중 급수 부등식}
    지수 $1 < p < \infty$와, $1/p + 1/q = 1$인 $q$를 고정하자.
    다음 부등식이 모든 $M, N$, $x \in L_M^p, y \in L_N^q$에 대해 성립한다.
    \begin{equation*}
        \left| \sum_m \sum_n \frac{x(m)y(n)}{m + n} \right| \leq \frac{\pi}{\sin(\pi/p)} \|x\|_p \|y\|_q
    \end{equation*}
\end{theorem}
\begin{proof}
    \Cref{thm:hdypolyaineq}에서, $K(m, n) = 1/(m + n)$을 사용하라.
\end{proof}

\section{지수의 볼록성}

먼저, 다변수함수 $f$의 볼록성을 고려하자.
\begin{definition}
\label{def:multiconvex} \index{convex function!multivariate} \index{볼록함수!다변수}
    유클리드 공간 $\mathbb{R}^d$의 볼록한 부분집합 $S$에 정의된 함수 $f: S \to \mathbb{R}$가 모든 $0 \leq t \leq 1$과 $x, y \in S$에 대해 $f(tx + (1 - t)y) \leq tf(x) + (1 - t)f(y)$가 성립하면, $f$를 볼록하다고 한다.
\end{definition}

다음은 볼록함수들의 일반적 성질 몇가지이다.

\begin{exercise}
    평면 $\mathbb{R}^2$에 정의된 함수 $|x|^2 = |x_1|^2 + |x_2|^2$가 볼록함을 보여라.
\end{exercise}
\begin{exercise}
    평면 $\mathbb{R}^2$에 정의된 함수 $f$가, 모든 $x, y \in \mathbb{R}$에 대해서 $z \mapsto f(x, z)$와 $z \mapsto f(z, y)$가 볼록함수라고 해서 $f$가 볼록함수일 필요는 없음을 보이시오.
\end{exercise}
\begin{exercise}
    두 실수 $a, b$가 양수이고, $f, g$가 같은 정의역 위 정의된 볼록함수이면, $af + bg$ 또한 볼록함수임을 보이시오.
\end{exercise}
\begin{exercise}
    유클리드 공간 $\mathbb{R}^d$에 대해 $0 \leq a_1, \dots, a_d$를 고정하고, $f : \mathbb{R}^d \to \mathbb{R}$을 $f(t) = a_1^{t_1} \dots a_d^{t_d}$로 정의하자.
    함수 $f$가 볼록함을 보이시오.
\end{exercise}

\begin{theorem}[Jensen]
\label{thm:multivariatejensen} \index{Jensen's inequality} \index{젠센 부등식}
    임의의 $0 \leq \lambda_i \leq 1, \sum \lambda_i = 1$에 대해서, $f$가 볼록하면, $f(\lambda_1 x_1 + \dots + \lambda_n x_n) \leq \sum \lambda_i f(x_i)$가 성립한다. (cf. \Cref{thm:Jensen})
\end{theorem}
\begin{exercise}
    \Cref{thm:multivariatejensen}을 증명하시오.
\end{exercise}

이제 \Cref{thm:youngconvineq}의 훨씬 좋은 증명을 보일 것이다.
먼저 합성곱의 (실질적으로 같은) $L^p$ 정의를 하자.
\begin{definition}
    두 $x\in L^p(\mu), y \in L^q(\mu)$에 대해 합성곱 $x * y \in L^r(\mu)$를 다음과 같이 정의한다.
    \begin{equation*}
        (x*y)(n) = \sum_m x(m) y(n - m) \mu(m)
    \end{equation*}
\end{definition}

\begin{exercise}
    모든 $x \in L^p(\mu), y \in L^q(\mu), z \in L^r(\mu)$에 대해 $x * y = y * x$와, $(x * y) * z = x * (y * z)$를 증명하시오.
    또한, $a, b \in \mathbb{C}$에 대해, $(ax + by) * z = a (x * z) + b (y * z)$를 증명하시오. [지금까지 그랬듯이, 증명에 큰 차이가 없을 것이다.]
\end{exercise}

\begin{lemma}
\label{lem:youngineq2}
    임의의 $N$에 대해서, $1 \leq i \leq N$에 각각 $x_i \in L^{p_i}_d(\mu)$를 대응하자.
    지수들 $1 \leq p_i \leq \infty$가 $\sum 1/p_i = N - 1$을 만족하면, $|(x_1 * x_2 * \dots * x_N)(0)| \leq \|x_1\|_{p_1} \dots \|x_N\|_{p_N}$가 성립한다.
\end{lemma}
\begin{proof}
    어느 $\|x_i\|_{p_i} = 0$이면, 자명하므로, 모두 $0$이 아니라고 가정하고, 각 $x_i$를 $\|x_i\|_{p_i}$로 나누면, 모든 $\|x_i\|_{p_i} = 1$이라고 가정 가능하다.
    또한 어느 $p_i = \infty$이면, 나머지 $p_j = 1$이므로, 이 경우는 완전히 자명하다. ($p_i = \infty$ 인 $x_k$를 횔더 부등식을 사용해 제거한 후, 모든 $p_j = 1$이므로, 귀납법과 횔더 부등식을 사용해라.)
    마지막으로, $y_i(n) = |x_i(n)|^{p_i}$, 그리고, $s_i = 1/p_i$로 정의하면, 모든 $y_i \in L^1_d(\mu), \|y_i\|_1 = 1$에 대해,
    \begin{equation}
    \label{eq:youngineq1}
        |(y_1^{s_1} * \dots * y_N^{s_N})(0)| \leq 1
    \end{equation}
    임을 증명하는 것으로 간단화 할 수 있다.
    여기서부터 $y_i$를 고정하자.
    
    모든 $0 \leq a_1, \dots, a_N$에 대해서, $t \in \mathbb{R}^N$에 대해 $f(t) = a_1^{t_1} \dots a_N^{t_N}$로 정의하면, (모든 $i$에 대해) $0 \leq t_i \leq 1$인 영역에서 $f$는 볼록이다.
    \Cref{eq:youngineq1}의 좌변을 $s \in \mathbb{R}^N$의 함수 $F(s)$로 보자.
    이 때 $F(s)$는 모두 $a_1^{s_1} \dots a_N^{s_N}$ 꼴의 항들의 합으로 표현되므로, 이 함수는 $0 \leq s_i \leq 1$인 영역에서 볼록이다.
    그런데
    \begin{equation*}
        \sum s_i = N - 1
    \end{equation*}
    이므로, $f_i$를 $i$번째 성분이 $0$이고 나머지 성분이 $1$인 $\mathbb{R}^N$의 원소로 놓으면
    \begin{equation}
    \label{eq:youngineq2}
        (s_1, \dots, s_N) = \sum (1 - s_i) f_i
    \end{equation}
    가 성립한다.
    \Cref{eq:youngineq2}의 식을 사용해 $F$의 볼록성을 (cf. \Cref{thm:multivariatejensen}) 사용하자.
    그러나 우리가 말했듯이 $F(f_i) \leq 1$이므로, ($p_i = \infty$인 경우에 대응된다.)
    \begin{equation*}
        F(s) \leq \sum (1 - s_i) F(f_i) \leq \sum (1 - s_i) \leq 1
    \end{equation*}
    에서 증명이 끝났다.
\end{proof}

\begin{theorem}[Young]
\label{thm:youngineq2} \index{Young's convolution inequality} \index{영의 합성곱 부등식}
    임의의 $N$에 대해서, $1 \leq i \leq N$에 각각 $x_i \in L^{p_i}_d(\mu)$이고, 모든 지수$1 \leq p_i \leq \infty$가 $\sum 1/p_i = N - 1 + 1/q$를 만족하면,
    \begin{equation*}
        \|x_1 * \dots * x_N\|_q \leq \|x_1\|_{p_1} \dots \|x_N\|_{p_N}
    \end{equation*}
    이다.
\end{theorem}
\begin{proof}
    먼저 지수 $q'$를 $1/q + 1/q' = 1$이도록 잡자.
    각 $x \in L^{q'}_d(\mu)$에 대해 $(Rx)(n) = x(d - n + 1)$으로 정의하면,
    \begin{equation*}
        (x_1 * \dots * x_N) * (Rx)(0) = \sum_n (x_1 * \dots * x_N)(n)x(n) \mu(n)
    \end{equation*}
    이고, $\|Rx\|_{q'} = \|x\|_{q'}$이다.
    
    \Cref{thm:holderconvlp}를 \Cref{lem:youngineq2}와, $(x_1 * \dots * x_N), x$에 적용하면, 원하는 결론이 나온다.
\end{proof}

\begin{exercise}
    \Cref{thm:youngineq2}과 같은 방법으로 일반화된 횔더 부등식, 즉, $1 \leq p_i, r \leq \infty$가 $\sum 1/p_i = 1/r$을 만족하면, 모든 $x_i \in L^{p_i}_d(\mu)$에 대해
    \begin{equation*}
        \|x_1 \dots x_N\|_r \leq \|x_1\|_{p_1} \dots \|x_N\|_{p_N}
    \end{equation*}
    를 만족함을 보이시오.
\end{exercise}

\Cref{thm:pnormint1}, \Cref{thm:hdypolyaineq}, 그리고 \Cref{lem:youngineq2}의 증명은 $\|\cdot\|_p$간의 ``연속적 보간''이 가능할 수도 있다는 인상을 준다.
이것이 실제로 가능하다는 놀라운 사실이 바로 다음 정리이다.

다음 정리에서도 변함 없이 모든 지수는 $[1, \infty]$의 범위로 본다.
\begin{theorem}[Riesz-Thorin]
\label{thm:rieszint} \index{Riesz-Thorin interpolation theorem} \index{리에즈-쏘린 보간 정리}
    선형 함수 $T: \mathbb{C}^d \to \mathbb{C}^{d'}$, 즉 모든 $c \in \mathbb{C}, x, y \in \mathbb{C}^d$
    \begin{align*}
        T(x + y) &= Tx + Ty \\
        Tcx &= cTx
    \end{align*}
    를 만족하는 $T$에 대해, $T$를 $L^{p_1}_d(\mu) \to L^{q_1}_{d'}(\mu')$로 본 함수를 $T_1$, $L^{p_2}_d(\mu) \to L^{q_2}_{d'}(\mu')$로 본 함수를 $T_2$라고 정의하자.
    그러면, 모든 $0 \leq t \leq 1$에 대해
    \begin{equation*}
        \left(\frac{1}{p}, \frac{1}{q}\right) = t \left(\frac{1}{p_1}, \frac{1}{q_1}\right) + (1 - t) \left(\frac{1}{p_2}, \frac{1}{q_2}\right)
    \end{equation*}
    이면, $L^p_d(\mu) \to L^q_d(\mu')$로 본 $T'$은 $\|T'\| \leq \|T_1\|^t \|T_2\|^{1 - t}$를 만족한다.
\end{theorem}

\begin{example}
    선형 함수 $L: \mathbb{C}^d \to \mathbb{C}^{d'}$을 $K(m, n) \in \mathbb{C}^{d \times d'}$, 즉 $1 \leq m \leq d, 1 \leq n \leq d'$에 정의되어 있는 $K$에 대해,
    \begin{equation*}
        (Lx)(n) = \sum_{m = 1}^d K(m, n)x(m)
    \end{equation*}
    으로 정의하고, $\sum_m |K(m, n)| \leq M_1, \sum_n |K(m, n)| \leq M_2$라고 하자.
    그러면,
    \begin{equation*}
        \|Lx\|_1 \leq \sum_{m = 1}^d M_2 |x(m)| = M_2\|x\|_1
    \end{equation*}
    이고,
    \begin{equation*}
        \|Lx\|_\infty \leq \max_n \left| \sum_{m = 1} K(m, n)x(m) \right| \leq \|x\|_\infty \max_n \left| \sum_{m = 1} K(m,n) \right| \leq \|x\|_\infty M_1
    \end{equation*}
    이므로,
    \begin{equation*}
        \|Lx\|_p \leq M_2^{1/p} M_1^{1 - 1/p} \|x\|_p
    \end{equation*}
    가 성립한다.
\end{example}

\begin{example}
    일반화된 횔더 부등식, 즉 $x_i \in L_{p_i}^d (\mu)$일 때,
    \begin{equation*}
        \sum \frac{1}{p_i} = \frac{1}{r}
    \end{equation*}
    이면,
    \begin{equation*}
        54
    \end{equation*}
\end{example}